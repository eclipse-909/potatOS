%%%%%%%%%%%%%%%%%%%%%%%%%%%%%%%%%%%%%%%%%
%
% CMPT 424N 111
% Fall Semester
% Lab 3
%
%%%%%%%%%%%%%%%%%%%%%%%%%%%%%%%%%%%%%%%%%

%%%%%%%%%%%%%%%%%%%%%%%%%%%%%%%%%%%%%%%%%
% Short Sectioned Assignment
% LaTeX Template
% Version 1.0 (5/5/12)
%
% This template has been downloaded from: http://www.LaTeXTemplates.com
% Original author: % Frits Wenneker (http://www.howtotex.com)
% License: CC BY-NC-SA 3.0 (http://creativecommons.org/licenses/by-nc-sa/3.0/)
% Modified by Alan G. Labouseur  - alan@labouseur.com
%
%%%%%%%%%%%%%%%%%%%%%%%%%%%%%%%%%%%%%%%%%

%----------------------------------------------------------------------------------------
%	PACKAGES AND OTHER DOCUMENT CONFIGURATIONS
%----------------------------------------------------------------------------------------

\documentclass[letterpaper, 10pt,DIV=13]{scrartcl} 

\usepackage[T1]{fontenc} % Use 8-bit encoding that has 256 glyphs
\usepackage[english]{babel} % English language/hyphenation
\usepackage{amsmath,amsfonts,amsthm,xfrac} % Math packages
\usepackage{sectsty} % Allows customizing section commands
\usepackage{graphicx}
\usepackage[lined,linesnumbered,commentsnumbered]{algorithm2e}
\usepackage{listings}
\usepackage{parskip}
\usepackage{lastpage}

\allsectionsfont{\normalfont\scshape} % Make all section titles in default font and small caps.

\usepackage{fancyhdr} % Custom headers and footers
\pagestyle{fancyplain} % Makes all pages in the document conform to the custom headers and footers

\fancyhead{} % No page header - if you want one, create it in the same way as the footers below
\fancyfoot[L]{} % Empty left footer
\fancyfoot[C]{} % Empty center footer
\fancyfoot[R]{page \thepage\ of \pageref{LastPage}} % Page numbering for right footer

\renewcommand{\headrulewidth}{0pt} % Remove header underlines
\renewcommand{\footrulewidth}{0pt} % Remove footer underlines
\setlength{\headheight}{13.6pt} % Customize the height of the header

\numberwithin{equation}{section} % Number equations within sections (i.e. 1.1, 1.2, 2.1, 2.2 instead of 1, 2, 3, 4)
\numberwithin{figure}{section} % Number figures within sections (i.e. 1.1, 1.2, 2.1, 2.2 instead of 1, 2, 3, 4)
\numberwithin{table}{section} % Number tables within sections (i.e. 1.1, 1.2, 2.1, 2.2 instead of 1, 2, 3, 4)

\setlength\parindent{0pt} % Removes all indentation from paragraphs.

\binoppenalty=3000
\relpenalty=3000

%----------------------------------------------------------------------------------------
%	TITLE SECTION
%----------------------------------------------------------------------------------------

\newcommand{\horrule}[1]{\rule{\linewidth}{#1}} % Create horizontal rule command with 1 argument of height

\title{	
   \normalfont \normalsize 
   \textsc{CMPT 424N 111 - Fall 2024 - Dr. Labouseur} \\[10pt] % Header stuff.
   \horrule{0.5pt} \\[0.25cm] 	% Top horizontal rule
   \huge Lab Three  \\     	    % Assignment title
   \horrule{0.5pt} \\[0.25cm] 	% Bottom horizontal rule
}

\author{Ethan Morton \\ \normalsize Ethan.Morton1@Marist.edu}

\date{\normalsize\today} 	% Today's date.

\begin{document}
\maketitle % Print the title

%----------------------------------------------------------------------------------------
%   start PROBLEM ONE
%----------------------------------------------------------------------------------------
\section{Problem One}
Q) Explain the difference between internal and external fragmentation.

A)\\
Processes are loaded and unloaded to and from memory all the time, so it naturally leads to a situation where processes are located sporadically with chunks of free memory between them. External fragmentation is the problem where processes cannot be allocated because despite there being enough free memory to allocate a process, those free chunks are each too small to contain the new process in a contiguous block of memory.

When the OS is allocating for a new process, it might find a free chunk of memory that sits between two other processes. This chunk might be just big enough to fit the new process, and there will be a matter of bytes or kilobytes left over in free memory. This leftover memory is not enough to allocate most processes, and it introduces overhead because the OS has to keep track of this tiny chunk of free memory, even though it will probably never use it. To reduce this overhead, the OS will typically include those remaining bytes or kilobytes in the new allocation, so it doesn't have to keep track of it. This memory has been allocated, but it won't ever be used by the process. Eventually the amount of this memory adds up, and this causes internal fragmentation. There is enough unused memory to allocate a new process, but it is not free memory.

%----------------------------------------------------------------------------------------
%   start PROBLEM TWO
%----------------------------------------------------------------------------------------
\section{Problem Two}
% For some reason it thought that 'f's were 'T's when I copied the question from the lab PDF,
% so I appologize if I missed any typos, especially where it says 'fit'.
Q) Given five (5) memory partitions of 100KB, 500KB, 200KB, 300KB, and 600KB (in that
order), how would optimal, first-fit, best-fit, and worst-fit algorithms place processes
of 212KB, 417KB, 112KB, and 426KB (in that order)?

A)\\
Syntax: process -> (is allocated to) memory partition

First-Fit:\\
1) 212KB -> 500KB\\
2) 417KB -> 600KB\\
3) 112KB -> 200KB\\
4) 426KB -> Cannot allocate

Best-Fit:\\
1) 212KB -> 300KB\\
2) 417KB -> 500KB\\
3) 112KB -> 200KB\\
4) 426KB -> 600KB

Worst-Fit:\\
1) 212KB -> 600KB\\
2) 417KB -> 500KB\\
3) 112KB -> 300KB\\
4) 426KB -> Cannot allocate

\end{document}